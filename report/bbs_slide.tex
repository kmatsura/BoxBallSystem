% \documentclass[dvipdfmx, 11pt]{beamer}
\documentclass[aspectratio=169, dvipdfmx, 11pt]{beamer} % aspectratio=43, 149, 169
\usepackage{here, amsmath, latexsym, amssymb, bm, ascmac, mathtools, multicol, tcolorbox, subfig}
\usepackage{braket}
%デザインの選択(省略可)
\usetheme{Luebeck}
%カラーテーマの選択(省略可)
\usecolortheme{orchid}
%フォントテーマの選択(省略可)
\usefonttheme{professionalfonts}
%フレーム内のテーマの選択(省略可)
\useinnertheme{circles}
%フレーム外側のテーマの選択(省略可)
\useoutertheme{infolines}
%しおりの文字化け解消
\usepackage{atbegshi}
\ifnum 42146=\euc"A4A2
\AtBeginShipoutFirst{\special{pdf:tounicode EUC-UCS2}}
\else
\AtBeginShipoutFirst{\special{pdf:tounicode 90ms-RKSJ-UCS2}}
\fi
%ナビゲーションバー非表示
\setbeamertemplate{navigation symbols}{}
%既定をゴシック体に
\renewcommand{\kanjifamilydefault}{\gtdefault}
%タイトル色
\setbeamercolor{title}{fg=structure, bg=}
%フレームタイトル色
\setbeamercolor{frametitle}{fg=structure, bg=}
%スライド番号のみ表示
%\setbeamertemplate{footline}[frame number]
%itemize
\setbeamertemplate{itemize item}{\small\raise0.5pt\hbox{$\bullet$}}
\setbeamertemplate{itemize subitem}{\tiny\raise1.5pt\hbox{$\blacktriangleright$}}
\setbeamertemplate{itemize subsubitem}{\tiny\raise1.5pt\hbox{$\bigstar$}}
% color
\newcommand{\red}[1]{\textcolor{red}{#1}}
\newcommand{\green}[1]{\textcolor{green!40!black}{#1}}
\newcommand{\blue}[1]{\textcolor{blue!80!black}{#1}}

\title[Quantinization of Preodic Box Ball System]{周期的境界条件における量子箱玉系}
%\subtitle{副題}
\author[k.matsuura]{松浦 健悟}
\institute[]{東京大学教養学部統合自然科学科物質基礎科学コース}
\date{\today}

\begin{document}
\maketitle


\begin{frame}{箱玉系とは...}
- 離散力学系の一種

- アーク則による時間発展の書き方
\leavevmode\\~\\
\includegraphics[width=14cm]{/home/kmatsuura/Desktop/4a/sotugyoukennkyuu/data/bbs_arkrule.png}
$$B_1 = \{0, 1\}, p \in B_1 ^ {\otimes L}, L \rightarrow \infty$$


\end{frame}

\begin{frame}{箱玉系とは...}

- carrierを使ったlocalな遷移\\
組み合わせR:$x \otimes b  \simeq \tilde{b} \otimes y(x,y \in B_\infty, b,\tilde{b} \in B_1)$
\leavevmode\\~\\
\leavevmode\\~\\
\leavevmode\\~\\
\includegraphics[width=14cm]{/home/kmatsuura/Desktop/4a/sotugyoukennkyuu/data/bbs_carrier_local.png}

\end{frame}

\begin{frame}{箱玉系とは...}

- 時間発展は組み合わせRを繰り返し適用したものとみなせる\\
\leavevmode\\~\\
\leavevmode\\~\\
\includegraphics[width=14cm]{/home/kmatsuura/Desktop/4a/sotugyoukennkyuu/data/bbs_carrier.png}
\end{frame}

\begin{frame}{周期的境界条件の適用}
右端と左端がつながるように境界条件を設定する。\\
時間発展は、キャリアを使って次のように表現できる。\\
\leavevmode\\~\\
\includegraphics[width=14cm]{/home/kmatsuura/Desktop/4a/sotugyoukennkyuu/data/bbs_carrier_preodic.png}
\leavevmode\\~\\
この箱の要素の並び$b_1 \otimes b_2 \otimes ... \otimes b_L$をパス$p$という。\\
周期箱玉系の時間発展は、 $m \otimes p \simeq T(p) \otimes m $と表すことができる。\\
周期的箱玉系の時間発展は周期軌道を形成する。

\end{frame}

\begin{frame}{量子化}
- 量子箱玉系の状態空間\\
\leavevmode\\~\\
量子箱玉系の任意の状態は、通常の箱玉系におけるパスの重ね合わせとして表す。\\
つまり、量子箱玉系の任意の状態$\ket{p}$は、 $a_i \in \mathbb{C} $として、通常の(箱の長さが無限の)箱玉系におけるパスにインデックスをつけたものである$\ket{p_mm}_i$を用いて、\\(冗長)
$$ \ket{p} = \sum_i{a_i\ket{p_mm}_i}$$
と表すことができる。\\

\end{frame}

\begin{frame}{量子箱玉系の時間発展演算子}
量子箱玉系の時間発展は古典系の組み合わせRに対応して、次のようなローカルな演算子を連続して作用させたものとして表せる。\\
$K_1 : $\\
\includegraphics[width=12cm]{/home/kmatsuura/Desktop/4a/sotugyoukennkyuu/data/qbbs_carrier_local.png}

$$K_1(i)([m] \otimes b_i)\simeq P_1[m] \otimes b_i + R_1[m] \otimes (b_i-1)$$
$$ L = ...K_1(1)K_1(0)K_1(-1)...$$
$$L(0 \otimes \ket{p}) \simeq T(\ket{p}) \otimes 0$$
 
\end{frame}

\begin{frame}{量子周期箱玉系の状態}
量子周期箱玉系の任意の状態は、通常の周期箱玉系におけるパスの重ね合わせとして表す。\\
例えば、$L=4,M=2$の系の場合、任意の状態$\ket{p}$は、$a_i \in \mathbb{C}$によって\\
$$\ket{p} = a_0\ket{1100} + a_1\ket{0110} + a_2\ket{0011} + a_3\ket{1001} + a_4\ket{1010} + a_5\ket{0101} $$
と表すことができる。
\end{frame}

\begin{frame}{量子周期箱玉系の時間発展の定義}
系の長さ$n$の量子周期箱玉系の時間発展を次のように定義した。
$$T(z)(\ket{p_mm} \rightarrow \ket{p'_mm}) = \sum_{m=0} ^ {\infty} z^{mn+\delta} \ket{p'_mm}_m$$
ただし、$\delta,\ket{p'_mm}_m$ は演算子Lによって次のように表される。\\
$$L: m \otimes \ket{p_mm} \simeq \ket{p'_mm} \otimes m$$
$$\delta = \sum_i (n-i)*(b_i - b'_i)$$
\end{frame}

\begin{frame}{量子周期箱玉系の時間発展の性質1}
$T(z)(\ket{b_1b_2...b_n} \rightarrow \ket{b'_1b'_2...b'_n})$は巡回シフトに対して普遍である。すなわち、次が成立。
$$T(z)(\ket{b_1b_2...b_n} \rightarrow \ket{b'_1b'_2...b'_n}) = T(z)(\ket{b_nb_1...b_{n-1}} \rightarrow \ket{b'_nb'_1...b'_{n-1}})$$

\end{frame}
\begin{frame}{量子周期箱玉系の時間発展の性質2}
$T(z),T(y)$は作用させるの$\ket{p}$の属する状態空間の基底が全て同じ周期軌道内に属する時に可換である。すなわち、次が成立。\\

$$[T(z),T(y)] \sum_i \ket{p_mm}_i = 0 \;\;\; (if \;\; p_i \in \Omega_p \;\;for \;\;all \;\;i )$$

ただし$\Omega_p$はpが属しているある単一の周期軌道を表す。\\

\end{frame}


\begin{frame}{}
\end{frame}

\end{document}
