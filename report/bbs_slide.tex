% \documentclass[dvipdfmx, 11pt]{beamer}
\documentclass[aspectratio=169, dvipdfmx, 11pt]{beamer} % aspectratio=43, 149, 169
\usepackage{here, amsmath, latexsym, amssymb, bm, ascmac, mathtools, multicol, tcolorbox, subfig}

%デザインの選択(省略可)
\usetheme{Luebeck}
%カラーテーマの選択(省略可)
\usecolortheme{orchid}
%フォントテーマの選択(省略可)
\usefonttheme{professionalfonts}
%フレーム内のテーマの選択(省略可)
\useinnertheme{circles}
%フレーム外側のテーマの選択(省略可)
\useoutertheme{infolines}
%しおりの文字化け解消
\usepackage{atbegshi}
\ifnum 42146=\euc"A4A2
\AtBeginShipoutFirst{\special{pdf:tounicode EUC-UCS2}}
\else
\AtBeginShipoutFirst{\special{pdf:tounicode 90ms-RKSJ-UCS2}}
\fi
%ナビゲーションバー非表示
\setbeamertemplate{navigation symbols}{}
%既定をゴシック体に
\renewcommand{\kanjifamilydefault}{\gtdefault}
%タイトル色
\setbeamercolor{title}{fg=structure, bg=}
%フレームタイトル色
\setbeamercolor{frametitle}{fg=structure, bg=}
%スライド番号のみ表示
%\setbeamertemplate{footline}[frame number]
%itemize
\setbeamertemplate{itemize item}{\small\raise0.5pt\hbox{$\bullet$}}
\setbeamertemplate{itemize subitem}{\tiny\raise1.5pt\hbox{$\blacktriangleright$}}
\setbeamertemplate{itemize subsubitem}{\tiny\raise1.5pt\hbox{$\bigstar$}}
% color
\newcommand{\red}[1]{\textcolor{red}{#1}}
\newcommand{\green}[1]{\textcolor{green!40!black}{#1}}
\newcommand{\blue}[1]{\textcolor{blue!80!black}{#1}}

\title[Quantinization of Preodic Box Ball System]{周期的境界条件における量子箱玉系}
%\subtitle{副題}
\author[k.matsuura]{松浦 健悟}
\institute[]{東京大学教養学部統合自然科学科物質基礎科学コース}
\date{\today}

\begin{document}
\maketitle


\begin{frame}{箱玉系とは...}

- アーク則による時間発展の書き方
\\~\\
\includegraphics[width=14cm]{/home/kmatsuura/Desktop/4a/sotugyoukennkyuu/data/bbs_arkrule.png}
\end{frame}

\begin{frame}{箱玉系とは...}

- carrierを使ったlocalな遷移\\
組み合わせR:$x \otimes b  \simeq \tilde{b} \otimes y(x,y \in B_\infty, b,\tilde{b} \in B_1)$
\\~\\
\\~\\
\\~\\
\includegraphics[width=12cm]{/home/kmatsuura/Desktop/4a/sotugyoukennkyuu/data/bbs_carrier_local.png}
\end{frame}

\begin{frame}{箱玉系とは...}

- 時間発展は組み合わせRを繰り返し適用したものとみなせる\\
\\~\\
\\~\\
\includegraphics[width=14cm]{/home/kmatsuura/Desktop/4a/sotugyoukennkyuu/data/bbs_carrier.png}
\end{frame}

\begin{frame}{周期的境界条件の適用}
右端と左端がつながるように境界条件を設定する。\\
時間発展は、キャリアを使って次のように表現できる。\\
\\~\\
\includegraphics[width=14cm]{/home/kmatsuura/Desktop/4a/sotugyoukennkyuu/data/bbs_carrier_preodic.png}
\\~\\
この箱の要素の並び$b_1 \otimes b_2 \otimes ... \otimes b_L$をパス$p$という。\\
周期箱玉系の時間発展は、 $m \otimes p \simeq T(p) \otimes m $と表すことができる。\\

\end{frame}

\begin{frame}{量子化}
\end{frame}

\begin{frame}{量子箱玉系の時間発展演算子}
\end{frame}

\begin{frame}{量子周期箱玉系1}
\end{frame}

\begin{frame}{量子周期箱玉系2}
\end{frame}

\begin{frame}{量子周期箱玉系3}
\end{frame}

\begin{frame}{結論}
\end{frame}












\begin{frame}{ブロック環境}
    \begin{block}{block}
    block
    \end{block}
    \begin{alertblock}{alertblock}
    alertblock
    \end{alertblock}
    \begin{exampleblock}{exampleblock}
    exampleblock
    \end{exampleblock}
    \begin{tcolorbox}[colframe=green,
    colback=green!10!white,
    colbacktitle=green!40!white,
    coltitle=black, fonttitle=\bfseries,
    title=My box]
        box contents
    \end{tcolorbox}
\end{frame}

\section{Section 2}
\begin{frame}{箇条書き}
    \begin{itemize}
    \item アイテム1
    \item \alert{アイテム2}
        \begin{itemize}
        \item アイテム1
        \item \alert{アイテム2}
            \begin{itemize}
            \item アイテム1
            \item \alert{アイテム2}
            \end{itemize}
        \end{itemize}
    \end{itemize}
    \[
    \bm{x}^\top\bm{y}
    \]
    \begin{enumerate}
    \item abcde
    \item \structure{ABCDE}
    \item 
    \end{enumerate}
\end{frame}

\end{document}
